\documentclass[10pt,a5paper]{article}

\PassOptionsToPackage{dvipsnames}{xcolor}
\usepackage[cochineal]{newtx} % font packages first
\usepackage{accsupp}
\usepackage{csquotes}
\usepackage{etoolbox}
\usepackage[margin=0.5in,right=1in,marginparwidth=1in]{geometry}
\usepackage{marginnote}
\usepackage{setspace}
\usepackage{pgffor}
\usepackage[listings,skins]{tcolorbox}
\usepackage{xcolor}
\usepackage[%
    colorlinks,
    linkcolor=blue,
    urlcolor=green!60!black,
]{hyperref}



\colorlet{keywordcolor}{Plum}
\DeclareTotalTColorBox{\terminal}{mO{default}}{%
    skin=bicolor,
    verbatim,
    colback=black!80,
    colupper=white,
    colframe=black,
    colbacklower=gray!10,
    boxsep=0.5em,
    middle=0em,
}{%
    \textcolor{VioletRed}{>}
    \lstinline[style=BashDark]|#1|
    \ifstrequal{#2}{default}{}{%
        \tcblower
        \small
        #2
    }
}
\DeclareTotalTCBox{\code}{m}{%
    verbatim,
    colback=black!20,
    colframe=black,
    boxsep=0.25em,
    boxrule=0em,
    arc=0em,
    left=0pt,
    right=0pt,
}{%
    \small\lstinline[style=BashLight]|#1|
}
\lstdefinestyle{BashDark}{%
    language=bash,
    keywordstyle=\color{cyan},
    stringstyle=\color{yellow},
    breaklines,
    prebreak={\textbackslash},
    %postbreak=\mbox{\BeginAccSupp{ActualText={}}\textcolor{BrickRed}{$\hookrightarrow$}\space\EndAccSupp{}},
}
\lstdefinestyle{BashLight}{%
    language=bash,
    %keywordstyle=\color{blue},
    %stringstyle=\color{red},
    %breaklines,
    %prebreak={\textbackslash},
    %postbreak=\mbox{\BeginAccSupp{ActualText={}}\textcolor{BrickRed}{$\hookrightarrow$}\space\EndAccSupp{}},
}
\newcommand{\keyword}[1]{%
    \textcolor{keywordcolor}{\bfseries #1}
    \marginnote{\parbox[1]{\marginparwidth}{\raggedright\footnotesize\bfseries\textcolor{keywordcolor}{#1}}}
}
\newcommand{\linux}{\software{Linux}}
\newcommand{\openfoam}{\software{OpenFOAM}}
\newcommand{\paraview}{\software{ParaView}}
\newcommand{\software}[1]{\textsf{#1}}
\newcommand{\windows}{\software{Windows}}
\setlength{\parskip}{0.5em}
\setlength{\parindent}{0em}



\begin{document}

\setstretch{1.1}
\sloppy

\title{A quick guide for getting started with \openfoam}
\author{Vachan Potluri}
\maketitle

\tableofcontents

\section{Introduction}
\label{sec:intro}
\openfoam{} is a software for performing numerical simulations.

\subsection{\openfoam{} is a software \ldots}
\label{sec:intro_subsec:software}
More specifically, it is a free and \keyword{open source} software. That means the underlying files that are used to develop the software are given to users. These \enquote{underlying files} are called the \keyword{source code}. Advanced users often modify this source code for their custom applications.

\par \openfoam{} doesn't have a graphical user interface (GUI). The user communicates with \openfoam{} through files and commands. This makes it less suitable for \windows{} users who are more familiar with GUI. However, \openfoam{} commands are very simple to use, and you can very quickly get familiar with them.

\par Since \openfoam{} does not have a GUI, it does not have a visualisation interface of its own, but is uses a software named \paraview{} for this purpose. \openfoam{} installs \paraview{} by itself without requiring any additional installation steps.

\subsection{\ldots for performing numerical simulation}
\label{sec:intro_subsec:simulation}
\openfoam{} can perform simulation of various kinds of physical problems: heat conduction, fluid flow, elastic deformation, combustion etc. \openfoam{} has different \keyword{solvers} for simulating each of these physical applications.

\par In addition to solvers, \openfoam{} also has some tools that are useful before and after a simulation. For example, if you want to simulate flow over a F1 race car, then \openfoam{} can also generate the mesh required for the simulation. After the simulation, it can calculate the net drag force on the car. Such tools are called \keyword{applications}.

\section{Installation}
\label{sec:installation}
These links contain instructions for installation of \openfoam{} (version 10) in detail.
\begin{itemize}
    \item \href{https://openfoam.org/download/10-ubuntu/}{Link for \software{Ubuntu}}.
    \item \href{https://openfoam.org/download/10-linux/}{Link for other \linux{} distributions}.
    \item \href{https://openfoam.org/download/windows/}{Link for \windows}. For \windows{}, \openfoam{} runs within an application called \enquote{Windows Subsystem for Linux} (WSL). WSL is a virtual environment that allows running \linux{} softwares on \windows{}. These video tutorials by J\'{o}zsef Nagy are very helpful for installation and getting familiar with WSL: \href{https://www.youtube.com/watch?v=w0bBOWlVlvA}{for \windows{} 10}, \href{https://www.youtube.com/watch?v=CeEJS1eT9NE}{for \windows{} 11}.
\end{itemize}

\section{The first simulation}
\label{sec:first_simulation}
First, create a new directory (folder) for performing simulations. We will call this the \enquote{run} directory.\\
\terminal{mkdir -p $FOAM_RUN}[\code{$FOAM_RUN} is an \enquote{environment variable} that is assigned by \openfoam{} during installation. The command \code{mkdir} (\enquote{make directory}) creates this directory.]

Now, copy a heat conduction tutorial case into this run directory.\\
\terminal{cp -r $FOAM_TUTORIALS/basic/laplacianFoam/flange $FOAM_RUN/}[\code{cp} (stands for \enquote{copy}) copies files from a source to a destination. \code{$FOAM_TUTORIALS} is a directory where many tutorial cases provided by \openfoam{} reside.]

Now navigate to this newly copied folder.\\
\terminal{cd $FOAM_RUN/flange}[\code{cd} (stands for \enquote{change directory}). This command sets the folder copied above as the working directory.]\\
You can notice that this folder contains an \software{ANSYS} mesh file with the name \code{flange.ans}. We will use this mesh for simulation. First, convert it into \openfoam's format.\\
\terminal{ansysToFoam flange.ans}[\code{ansysToFoam} is an \openfoam{} application for reading \software{ANSYS} meshes.]

\code{laplacianFoam} is the solver name for heat conduction simulation. Execute the solver to perform the simulation.\\
\terminal{laplacianFoam}

Now, to visualise the results.\\
\terminal{paraFoam}[This launches the \paraview{} software and also opens the results in the current directory.]
\end{document}
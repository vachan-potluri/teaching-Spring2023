\documentclass[%
    10pt,
    xcolor={dvipsnames},
    % handout,
]{beamer}
\usepackage{amsmath}
\usepackage{amssymb}
\usepackage[%
    isbn=false,
]{biblatex}
\usepackage{lstautogobble}
\usepackage[listings,skins,theorems,xparse]{tcolorbox}
\usepackage{siunitx}
\usetheme[miniframes,framenumber-footline]{Mumbai} % available at https://github.com/vachan-potluri/beamer_themes/tree/main/Mumbai



\DeclareTotalTCBox{\inlinecode}{m}{%
    verbatim,
    size=tight,
    boxsep=2pt,
    arc=0pt,
    bottomrule=0pt,
    toprule=0pt,
    leftrule=0pt,
    rightrule=0pt,
}{\lstinline|#1|} % for printing inline code
\lstdefinestyle{ScilabStyle}{%
    % scilab formatting style
    language=Scilab,
    backgroundcolor=\color{gray!10},
    basicstyle=\linespread{1.25}\ttfamily\small,
    commentstyle=\color{green!70!black},
    keywordstyle=\bfseries\color{blue},
    numberstyle=\color{gray}\sffamily\footnotesize,
    stringstyle=\color{red},
    basewidth=0.6em,
    breakatwhitespace=false,
    breaklines=true,
    postbreak=\hbox{$\hookrightarrow$},
    captionpos=t,
    frame=l,
    keepspaces=true,
    numbers=none,
    numbersep=5pt,
    showspaces=false,
    showstringspaces=false,
    showtabs=false,
    tabsize=2,
    morekeywords=[2]{\%t,\%f,\%pi,\%e,\%i,\%inf,\%nan,\%eps},
    keywordstyle=[2]\color{magenta},
    morecomment=[s][\bfseries\color{orange!70}]{/*}{*/},
    %morecomment=[l][\itshape\color{blue!30!green}]{//-},
    autogobble=true,
    xleftmargin=1em,
    xrightmargin=1em,
}
\lstset{style=ScilabStyle}
\newcommand{\email}[1]{\href{mailto:#1}{\texttt{#1}}} % for email ids
\newcommand{\hint}[1]{{\small\alert{Hint: #1}}}
\newcommand{\keyword}[1]{#1}
\newcommand{\matlab}{\texttt{MATLAB}}
\newcommand{\scilab}{\texttt{Scilab}}
\newcommand{\scinotes}{\texttt{SciNotes}}
\newcommand{\setitemsep}[1]{\setlength\itemsep{#1}}
\newtcbox{\key}{%
    enhanced,
    on line,
    colback=gray!25,
    size=fbox,
    bottomrule=0pt,
    toprule=0pt,
    leftrule=0pt,
    rightrule=0pt,
    fontupper=\ttfamily,
    drop fuzzy shadow,
} % for printing keyboard keys
\newtcolorbox[auto counter]{exercise}{%
    code={
        \usebeamercolor{block title}
        \colorlet{titlebg}{bg}
        \colorlet{titlefg}{fg}
        \usebeamercolor{block body}
        \colorlet{bodybg}{bg}
        \colorlet{bodyfg}{fg}
    },
    colbacktitle=titlebg,
    coltitle=titlefg,
    colback=bodybg,
    coltext=bodyfg,
    size=fbox,
    title={Exercise~\thetcbcounter},
}% boxed env for exercises




\title{Scilab}
\author[Vachan Potluri]{%
    Vachan Potluri\\
    \email{vachanpotluri@iitb.ac.in}
}
\addbibresource{references.bib}



\begin{document}

\setlength\leftmargini{1em} % left margin, first level itemize
\setlength\leftmarginii{\leftmargini} % left margin, 2nd level
\setlength\leftmarginiii{\leftmarginii} % left margin, 3rd level

{
    \setbeamertemplate{headline}{}
    \setbeamertemplate{footline}{}
    \begin{frame}
        \titlepage
    \end{frame}
}

\begin{frame}{Introduction}
    \begin{block}{What is \scilab?}
        A free alternative to \matlab
    \end{block}
    \begin{block}<visible@2->{What can it do?}
        \begin{enumerate}
            \item Advanced calculator
            \item Programming
            \item Plotting, visualisation
        \end{enumerate}
    \end{block}
\end{frame}

\section{\scilab{} console}
\subsection{Advanced calculator}
\begin{frame}[fragile]{Simple calculations}
    Try out these and see if they give expected results
    \begin{lstlisting}[numbers=left]
        2+3-4
        4^2
        4**4
        6/4
        2+(2^2-(1/2))
        1e-3 + 1d-2
    \end{lstlisting}
\onslide<2->
    See what happens when you add a semicolon
    \begin{lstlisting}
        6/4;
    \end{lstlisting}
\end{frame}

\begin{frame}[fragile]{Variables}
    All calculations are stored by default in \inlinecode{ans}
    \begin{lstlisting}
        6/4;
        ans
    \end{lstlisting}
\onslide<2->
    You can specify a variable to store the value instead
    \begin{lstlisting}
        pi_approx = 22/7;
    \end{lstlisting}
    And see its value later
    \begin{lstlisting}
        pi_approx
        disp(pi_approx)
    \end{lstlisting}
\end{frame}

\begin{frame}[fragile]{More on variables}
    Some useful pre-defined variables
    \begin{lstlisting}[numbers=left]
        %pi
        %e
        %i
        %t
        %f
        %inf
        %nan
        %eps
    \end{lstlisting}
\end{frame}

\begin{frame}[fragile]{Pre-defined functions}
    See if the outputs of these lines are as expected
    \begin{lstlisting}[numbers=left]
        abs(-2)
        min(3,4,5)
        max(-2,-3,-4)
        sin(%pi/2)
        cos(%pi)
        tan(%pi/4)
        asin(1)/(%pi/2)
        exp(2)/%e^2
        log10(100)
        log(%e)
    \end{lstlisting}
    Auto-completion: hit \key{TAB}
\end{frame}

\begin{frame}[fragile]{Other \scilab{} windows}
\begin{itemize}
    \item \keyword{Variable Browser}
    \begin{itemize}
        \item Only lists user-defined variables
        \item To list all variables:
        \begin{lstlisting}
            whos
        \end{lstlisting}
        \onslide<2->
        \item You can delete all or specific user-defined variables
        \begin{lstlisting}
            pi_approx = 22/7;
            disp(pi_approx)
            clear pi_approx
            disp(pi_approx)
        \end{lstlisting}
    \end{itemize}
    \onslide<3->
    \item \keyword{Command History}
    \begin{itemize}
        \item Execute an old command by double clicking
        \item Can also navigate using $\uparrow$ and $\downarrow$ keys
        \item Clear screen using \inlinecode{clc}
    \end{itemize}
    \onslide<4->
    \item \keyword{File Browser}
    \begin{itemize}
        \item Useful when working with multiple files
    \end{itemize}
\end{itemize}
\end{frame}

\subsection{Arrays and matrices}
\begin{frame}[fragile]{Basic matrix creation}
    Wrap inside \inlinecode{[]}, use \inlinecode{,} and \inlinecode{;} to separate columns and rows
    \begin{lstlisting}
        x = [1,2,3]
        y = [4;5;6;7]
        A = [1,0;0,1]
    \end{lstlisting}
\onslide<2->
    \scilab{} will warn you if the dimensions are inconsistent
    \begin{lstlisting}
        B = [1,2,3;4,5]
    \end{lstlisting}
\onslide<3->
    Adding \inlinecode{'} will transpose the matrix
    \begin{lstlisting}
        B = [1,2,3;4,5,6];
        B'
    \end{lstlisting}
\onslide<4->
    You can fill matrices with pre-existing matrices
    \begin{lstlisting}
        row1 = [1,2,3,4];
        row2 = [5,6,7,8];
        M = [row1;row2]
    \end{lstlisting}
\end{frame}

\begin{frame}[fragile]{Special functions for matrix creation}
    Creating ranges
    \begin{lstlisting}
        i = 1:10
        j = 1:2:10
        x = 0:0.1:1
        y = linspace(0,1,25)
    \end{lstlisting}
\onslide<2->
    Some useful commands for creating dummy matrices of required size
    \begin{lstlisting}
        A = zeros(2,2)
        B = ones(3,2)
        M = eye(3,3)
    \end{lstlisting}
\onslide<3->
    Can you make sense of this result?
    \begin{lstlisting}
        M = [[zeros(1,2);ones(1,2);eye(2,2)],ones(4,1)]
    \end{lstlisting}
\end{frame}

\begin{frame}[fragile]{Matrix operations}
    \begin{columns}
        \begin{column}{0.49\linewidth}
            Scalar operations affect all elements of matrices
            \begin{lstlisting}
                A = eye(3,3);
                A*2
                A/4
                A+5
            \end{lstlisting}
\onslide<2->
            \scilab{} automatically figures out matrix operations too
            \begin{lstlisting}
                B = 2*ones(3,3)
                A+B
                A*B
                B^2
            \end{lstlisting}
        \end{column}
        \begin{column}{0.49\linewidth}
\onslide<3->
            Special element wise operations
            \begin{lstlisting}
                A.+B
                A.*B
                A.^B
                A./B
                A.^2
            \end{lstlisting}
            How is \inlinecode{A^2} different from \inlinecode{A.^2}?
        \end{column}
    \end{columns}
\end{frame}

\begin{frame}[fragile]{Matrix functions}
    Most \scilab{} functions can operate element-wise on matrices
    \begin{lstlisting}
        A = %pi/2*[0,1;2,3];
        sin(A)
    \end{lstlisting}
\onslide<2->
    Some special functions for matrices
    \begin{lstlisting}
        length(A)
        size(A)
        sum(A)
        det(A)
        inv(A)
        trace(A)
    \end{lstlisting}
\end{frame}

\begin{frame}[fragile]{Matrix indexing}
    \begin{columns}
        \begin{column}{0.49\linewidth}
            Access elements using \inlinecode{(row,col)}
            \begin{lstlisting}
                A = eye(3,3);
                A(1,2) = 2;
                A
            \end{lstlisting}
\onslide<2->
            A single index can also be used: increments column-wise
            \begin{lstlisting}
                A(4)
            \end{lstlisting}
\onslide<3->
            Extract rows and columns using \inlinecode{:}
            \begin{lstlisting}
                A(:,2)
                A(1,:)
            \end{lstlisting}
\onslide<4->
            Special symbol \inlinecode{\$}
            \begin{lstlisting}
                A($,3)
            \end{lstlisting}
        \end{column}
        \begin{column}{0.49\linewidth}
\onslide<5->
            Arrays can also be used to access and modify
            \begin{lstlisting}
                A([1,2],2)
                A(4,:) = [10,20,30]
            \end{lstlisting}
\onslide<6->
            See if this makes sense
            \begin{lstlisting}
                A = eye(4,4);
                j = [2,4];
                A(1,j) = j
                A([7,8]) = 50
                A($,$) = -1
                B = [9,10;j];
                A(B) = 100
            \end{lstlisting}
        \end{column}
    \end{columns}
\end{frame}

\subsection{Miscellaneous}
\begin{frame}[fragile]{Strings}
    Wrap in \inlinecode{""} or \inlinecode{''}
    \begin{lstlisting}
        fname = "Vachan";
        lname = 'Potluri';
        fname + lname
    \end{lstlisting}
\onslide<2->
    Function \inlinecode{string} converts variables to strings
    \begin{lstlisting}
        A = eye(2,2)
        string(A)
    \end{lstlisting}
\end{frame}

\begin{frame}[fragile]{Saving and loading data}
    \scilab{} has a working directory
    \begin{lstlisting}
        pwd
    \end{lstlisting}
    Working directory can be changed from File Browser (and also using \inlinecode{cd} or \inlinecode{chdir})\\[0.5em]
\onslide<2->
    Function \inlinecode{save} saves user-defined variables to a file in working directory
    \begin{lstlisting}
        x = 1.5;
        A = [1,2;3,4]
        save("data.dat")
    \end{lstlisting}
\onslide<3->
    These variables can be loaded for use later
    \begin{lstlisting}
        listvarinfile("data.dat")
        load("data.dat")
    \end{lstlisting}
\end{frame}

\begin{frame}[fragile]{Accessing help}
    \scilab's built-in help functionality is very useful
    \begin{lstlisting}
        help
        help save
    \end{lstlisting}
\end{frame}

{
% modify ',' to allow math mode line break at ','
% https://tex.stackexchange.com/a/67540/133968
\def\OldComma{,}
\catcode`\,=13
\def,{%
    \ifmmode%
    \OldComma\discretionary{}{}{}%
    \else%
    \OldComma%
    \fi%
}%
\sisetup{list-separator={,~}}
\begin{frame}{Exercises\footfullcite{amos2017}}
    \begin{columns}
        \begin{column}{0.49\linewidth}
            \begin{exercise}
                The pressure drop $\Delta p$ required for a flow rate $Q$ in a pipe of diameter $D$ is
                \begin{equation*}
                    \Delta p = 4.52 \frac{Q^{1.85}}{C^{1.7} D^{4.87}}
                \end{equation*}
                Find $\Delta p$ for these combinations of flow rates and diameters:
                \begin{itemize}
                    \item $Q=\numlist{50;100;200;400;1000}$
                    \item $D=\numlist{0.5;1;1;2;4}$
                \end{itemize}
            \end{exercise}
        \end{column}
        \begin{column}{0.49\linewidth}
\onslide<2->
            \begin{exercise}
                A magic square is a matrix in which all rows, columns and diagonals sum to same number.
                \begin{enumerate}
                    \item Generate a magic square of size 10
                    \item Verify its properties
                \end{enumerate}
                \hint{search \scilab{} help for the function \inlinecode{testmatrix}}
            \end{exercise}
        \end{column}
    \end{columns}
\end{frame}
}

\section{Programming}
\begin{frame}{\scinotes: built-in editor}
    \begin{itemize}
        \setitemsep{1em}
        \item Console is only useful for short calculations
        \item A single file containing all commands is useful for large calculations
        \item<2-> \scilab{} can do this through ``scripts'' or ``executables''
        \item<3-> \scinotes{} is \scilab's builtin-in GUI for handling scripts
        \item<4-> Customary to save such files with \inlinecode{.sce} extension
    \end{itemize}
\end{frame}

\begin{frame}[fragile]{Conditional statements}
    Can you make sense of this?
    \begin{lstlisting}
        x=6;
        reminder = modulo(x,3);
        
        if reminder==0 then
            disp("3 divides x")
        elseif reminder==1 then
            disp("x leaves reminder 1 when divided by 3")
        else
            disp("x leaves reminder 2 when divided by 3")
        end
    \end{lstlisting}
    \hint{look at help for function \inlinecode{modulo}}\\[1em]
\onslide<2->
    Logical expressions generally use \inlinecode{==, ~=, <, <=, >, >=, &&, \|\|, \%t, \%f}
\end{frame}

\begin{frame}[fragile]{Loops}
    \begin{columns}
        \begin{column}{0.5\linewidth}
            \begin{lstlisting}[numbers=left]
                array = 1:10;
                value = 5;
                
                for a=array
                    if value==a then
                        disp("Value exists in array");
                        break;
                    end
                end
            \end{lstlisting}
            What does \inlinecode{break} statement do?
        \end{column}
        \begin{column}{0.49\linewidth}
            \onslide<2->
            \scilab{} always loops over columns
\onslide<3->
            \begin{lstlisting}[numbers=left]
                array=[1;2;3]
                i=1;
                for a=array
                    disp("Element " + string(i) + ":")
                    disp(a)
                    i = i+1;
                end
            \end{lstlisting}
        \end{column}
    \end{columns}
\end{frame}

\begin{frame}[fragile]{Functions}
    \begin{lstlisting}
        function [Tf,Tk] = centigradeToFarenhietKelvin(Tc)
            Tf = Tc*9/5 + 32;
            Tk = Tc + 273;
        endfunction
        
        [Tf,Tk] = centigradeToFarenhietKelvin(37);
        disp(Tf)
        disp(Tk)
    \end{lstlisting}
    Here \inlinecode{Tf} and \inlinecode{Tk} are the ``return'' values; \inlinecode{Tc} is the parameter\\[0.5em]
\onslide<2->
    Can also have multiple parameters
    \begin{lstlisting}
        function s = sum(a,b)
            s = a+b;
        endfunction
        disp(sum(1,2));
    \end{lstlisting}
\end{frame}

\end{document}